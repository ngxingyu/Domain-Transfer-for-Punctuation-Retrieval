\documentclass[a4paper]{article}
\usepackage{commath}
\usepackage[utf8x]{inputenc}
\usepackage[left=1.5in,right=1.5in,top=1in,bottom=1in]{geometry}
\usepackage{verbatim} %comments
\usepackage{natbib}
\bibliographystyle{apalike}

\usepackage{hyperref}
\title{Review of research conducted for Punctuation Retrieval.}
\author{Ng Xing Yu}
\date{}
\begin{document}
\maketitle

\section{Introduction}
\label{introduction}
Punctuation Retrieval is an important aspect in any Automatic Speech Recognition (ASR) pipeline for two reasons: to improve readability of auto-generated transcripts for videos or podcast subtitling or voice dictation applications, and to better capture the meaning of speech transcripts to improve the performance of downstream Natural Language Processing (NLP) tasks.

This review will look into the ideas taken into account by different authors and discuss potential areas of exploration to improve ASR output.

\section{Punctuation Features}
The majority of research into punctuation retrieval on English speech transcripts condense all punctuation into four classes --- (Period .), (Comma ,), (Question Mark ?) and (None), using a custom-defined mapping function to replace other punctuation with the four classes. This is done to combat the issue of an imbalance in punctuation occurrence in most datasets, with less frequent punctuation like semicolons or dashes occurring under 1\% in the entire \href{http://opus.nlpl.eu/OpenSubtitles-v2016.php}{OpenSubtitles v2016 english corpus}. An earlier paper by \cite{dynamiccrf} included the (Exclamation mark !) class, but did not comment on the performance of prediction on the less common classes. 
The paper by \cite{birnnattention} also used a different mapping scheme for the two languages evaluated --- Estonian and English, with the following differences:
\begin{center}
\begin{tabular}{|c|c|c|}
\hline
 &\multicolumn{2}{|c|}{Mapped to} \\
 \hline
From & Estonian & English \\
\hline
; ! & Period . & Period . \\
: & Period . & Comma , \\
- --- & None & Comma ,\\
\hline
\end{tabular}
\end{center}
While these mappings are logical, with (?) and (!) marking sentence boundaries and the emdash (---) indicating a break in sentence structure, the mapping would result in some meaning and context being lost. Hence, expanding the set of punctuation classes to all punctuation which can be inferred from a speech, for instance \{! , - . : ; ? --- \textellipsis "\} for English, would improve the readibility of the ASR output and allow the text to better represent the meaning within the audio input.
\section{Data}
The data sources for training punctuation retrieval tasks can be categorised into purely textual, and aligned audio with text.
The most common source used for training punctuation retrieval is the IWSLT dataset - many versions of which features transcripts from TED-talks and OpenSubtitles. This textual datasource is used to train a model that only uses lexical features for punctuation retrieval. \cite{jointlearningcorrbirnn} also used transcripts from the Intelligence squared debate show. \cite{medicalasr} used an internal medical speech transcript dataset to demonstrate the ability of their model to transfer to a domain with large vocabulary and data scarcity.

\textbf{Use of non-lexical features} \cite{multimodalsemi} used the audio and text data from the Fisher corpus in training a model using both acoustic and lexical features. \cite{adversarial} also made use of the Penn Treebank-3 dataset to train a POS tagger to improve the performance of the punctuation retrieval class. \cite{birnnattention} made use of Estonian text annotated with pause duration (e.g. $\langle sil=0.030\rangle$ for 30 milliseconds of silence) 

\textbf{Data Augmentation} In performing ASR, the generated text from speech might contain word errors in the form of insertions, deletions, and substitutions, which would affect the accuracy of the downstream punctuation retrieval task. To make the model more robust to such word errors, \cite{noisy} simulated the three forms of word errors, inserting or substituting existing tokens with the unknown token and randomly deleting tokens. Their RoBERTa-large model obtained higher F1 scores on the test set --- ASR output generated by the Google Cloud Speech API to simulate word errors --- when trained on augmented text, demonstrating the effectiveness of introducing noise into the training data in improving the model's robustness to word errors. 

\cite{speechtranslationrobust} presented four strategies to generate noise for ASR training: Substitution with placeholder, substitution with random token, substitution with more frequent character, and substitution with homophone based character. Being trained and evaluated on a Chinese corpus, many homophonous words have different meaning. The proposed strategies proved effective in improving speech to text translation and can be adapted to the field of punctuation retrieval.

Other possible noise injection techniques include splitting of words into subwords or homophone or random substitution/insertion/deletion at the subword level.

\section{Models}
There are two categories of models that have been applied to the task of punctuation retrieval: Recurrent Neural Networks and feed-forward networks.
The current state of the art models as of 2020 features a BERT (variant) base layer with a variety of layer combinations above it. These models take a sequence of text represented as tokens or word embedding as input and output a sequence of corresponding punctuation labels, similar to other sequence tagging tasks like Part-of-speech (POS) or Named-Entity Recognition (NER) tagging.
\subsection{Conditional Random Fields (CRF)}
\cite{dynamiccrf} adopts a multi-task approach, training the model on punctuation prediction and sentence boundary labelling. They demonstrate the effectiveness of the factorial-CRF in improving performance of punctuation prediction, by learning a sentence boundary tag (e.g. start of question sentence or within declarative sentence) along with the punctuation tags.

\cite{mecrf} evaluates the ability of various CRF models to capture long-range dependencies effectively, to better represent the input text. Their proposed Memory-Enhanced CRF takes inspiration from \cite{memnet}, and features a memory layer ???
\subsection{Recurrent Neural Networks (RNN) / Multi-layered RNN}
Prior to the release of the BERT models, BiLSTMs were among the best performing models for NLP, being able to utilise information from both sides to predict the tokens of the output sequence. The introduction of gates within the Long short-term memory (LSTM) or Gated recurrent unit (GRU) cells allows the model to preserve information across time-steps, giving the model the ability to capture long-term dependencies.  
\cite{birnnattention} demonstrates the effectiveness of bi-directionality in improving the performance of punctuation retrieval, as punctuation relies on cues from words in both directions. They also observed a slight improvement in restoration of question-marks when an attention mechanism is added to determine the relative importance of each BiLSTM output in determining each output label.

\cite{kim_2019} built upon this research, utilising a Deep-RNN model with a multi-head attention mechanism as proposed by \cite{attentionisallyouneed}. It relies on the stacked RNNs to extract sequential context before using the multi-head attention to focus on the context at each time-step. The use of multiple RNN layers would allow the model to learn various representations of the sequence. Based on the results obtained, there is a positive relationship between the number of RNN layers and the F1 score of the model. The paper does not provide any source code or details regarding the training process. Even so, the sequential nature of the model and its depth makes the model harder to tune and takes longer to train as compared to a single BiLSTM layer and possibly even a BERT model. Further research has to be conducted to evaluate the efficiency and performance of utilising the stacked RNN layers as compared to the BERT model.

\subsection{Pre-trained models}
\subsubsection{GloVe}
The use of word embeddings pre-trained using a large generic corpus (e.g. wikipedia or common crawl) allows the model to perform better on unseen examples as words with similar meaning would have a closer representation, allowing the model to generalise on unseen examples.
Both \citet{birnnattention} and \citet{kim_2019} obtained visible F1 score improvements when using pre-trained GloVe \citep{glove} embeddings as compared to randomly initialised word embeddings. 
\subsubsection{BERT}
The BERT model features is pre-trained on two tasks: Masked Language modelling (MLM) and Next Sentence Prediction (NSP), which allows the model to learn bidirectional representations.
The attention mechanism captures sequential information using positional encodings, represented by sine and cosine functions of different frequencies: 

$PE_{(2i)} = \sin{(pos/10000^{2i/d_{\text{model}}})}$

$PE_{(2i+1)} = \cos{(pos/10000^{2i/d_{\text{model}}})}$. 

This 

\subsection{BERT-(BiLSTM)-CRF}
A common model used in many papers feature the BERT-BiLSTM-CRF. However, there is little evidence of the effectiveness of adding a BiLSTM layer to increase the performance of the tagger. \cite{bertcrf} performed a comparison between different combinations of BERT with BiLSTM and CRF, and the addition of the BiLSTM layer led to a drop in performance of the model. \cite{chinesebertbilstm} presented the BERT-BiLSTM and BERT-BiLSTM-CRF model but not the BERT-CRF or BERT model, providing no information about the effectiveness of the BiLSTM layer.

\subsection{Multi-task learning}
Punctuation with POS-tagger, adversarial learning

Capitalization

Sentence Boundary \cite{dynamiccrf}

Use of acoustic features


\section{Training process}
\subsection{Chunking}
\subsection{Dealing with class imbalance}
There are various possible approaches to dealing with imbalance across punctuation classes. The bulk of research conducted deals with this class imbalance by absorbing the minority classes into a more generic class like period or comma. This is feasible as most punctuation can be generalised into two classes --- periods for sentence boundaries and commas for separating different parts of a sentence. However, this simply avoids the problem of class imbalance with punctuation like (?) being underrepresented. Dealing with class imbalance can improve the performance on weaker classes like (?) or (!) and can even allow for the training of a system that can retrieve other classes of punctuation like (``") or (---). 
\subsubsection{Weighted Cross-entropy (W-CEL)}
The multiplication of the component losses by a weight inversely proportional to the class frequency within the corpus will allow the model to converge faster on the weaker classes. A general formula for this loss function is given by: \[
J_{wcel}=-\frac{1}{M}\sum_{k=1}^{K}\sum_{m=1}^{M}w_k \times y_m^k \times log(f(x_m))
\] where \newline
$w_k$ represents the class weights, often taken to be the reciprocal of the class proportion within the training data \newline
$M$ represents the number of training examples \newline
$K$ represents the number of classes.
This approach was utilised by \cite{adaptivenerunbalanceddata} in his binary classifier to identify weaker classes. \citet{efficientbertrobust} mentions that their implementation of focal loss or class weights did not outperform the generic cross-entropy loss. This might be due to the use of just three punctuation classes with only a slightly weaker (?) class, making any gains less pronounced. The lack of parameter tuning in their experiment may also reduce the performance of the models using focal loss or class weights.

\subsubsection{Sørensen–Dice coefficient / F1 Score}
\citet{li2020dice} proposed the use of a self-adjusting dice loss as an alternative to cross-entropy loss. The Sørensen–Dice coefficient (DSC): $DSC(A,B)=\frac{2 \times\abs{A \cap B}}{\abs{A} + \abs{B}}$ aims to maximise the F1 score of the model. Their proposed loss function introduces a decaying factor to the DSC function to increase the importance of less confident classes and allowing the training process to focus on those classes. Their proposed loss function is as follows: \[
DSC(x_i)=\frac{2{(1-p_{i1})}^{\alpha}p_{i1}\cdot y_{i1}+\lambda}{{(1-p_{i1})}^{\alpha}p_{i1}+ y_{i1}+\lambda}\], with $p$ being the probability of a token being assigned to a label $i$, and $y$ representing the actual label of the token matching label $i$.
The model trained with the DSC loss achieved a higher F1 score across all evaluated tasks as compared to Focal Loss and Dice Loss. The results obtained when trained on datasets of varying levels of imbalance further supported the hypothesis that the DSC loss is effective in combating the issue of class imbalance. While providing a breakdown of scores per class would be most effective in demonstrating the impact of using DSC on the weaker classes, the results presented makes the use of DSC worth considering as a loss function in punctuation retrieval.


% \section*{}

\bibliography{asr.bib}

\end{document}