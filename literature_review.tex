\documentclass[12px,a4paper]{article}
\usepackage[utf8x]{inputenc}
\usepackage[left=1.5in,right=1.5in,top=1in,bottom=1in]{geometry}
\usepackage{verbatim} %comments
\usepackage{natbib}
\bibliographystyle{apalike}

\usepackage{hyperref}
\title{Review of research conducted for Punctuation Retrieval.}
\author{Ng Xing Yu}
\date{}
\begin{document}
\maketitle

\section{Introduction}
\label{introduction}
Punctuation Retrieval is an important aspect in any Automatic Speech Recognition (ASR) pipeline for two reasons --- to improve readability of auto-generated transcripts for videos or podcast subtitling or voice dictation applications, and to better capture the meaning of speech transcripts to improve the performance of downstream Natural Language Processing (NLP) tasks.

This review will look into the ideas taken into account by different authors and discuss potential areas of exploration to improve ASR output.

\section{Punctuation Features}
The majority of research into punctuation retrieval on English speech transcripts condense all punctuation into four classes --- (Period .), (Comma ,), (Question Mark ?) and (None), using a custom-defined mapping function to replace other punctuation with the four classes. This is done to combat the issue of an imbalance in punctuation occurrence in most datasets, with less frequent punctuation like semicolons or dashes occurring under 1\% in the entire \href{http://opus.nlpl.eu/OpenSubtitles-v2016.php}{OpenSubtitles v2016 english corpus}. An earlier paper by \cite{dynamiccrf} included the (Exclamation mark !) class, but did not comment on the performance of prediction on the less common classes. 
The paper by \cite{birnnattention} also used a different mapping scheme for the two languages evaluated --- Estonian and English, with the following differences:
\begin{center}
\begin{tabular}{|c|c|c|}
\hline
 &\multicolumn{2}{|c|}{Mapped to} \\
 \hline
From & Estonian & English \\
\hline
; ! & Period . & Period . \\
: & Period . & Comma , \\
- & None & Comma ,\\
\hline
\end{tabular}
\end{center}
While these mappings are logical, with (?) and (!) marking sentence boundaries and the emdash (---) indicating a break in sentence structure, the mapping would result in some meaning and context being lost. Hence, expanding the set of punctuation classes to all punctuation which can be inferred from a speech, for instance {! , - . : ; ? — … "} for English, would improve the readibility of the ASR output and allow the text to better represent the meaning within the audio input.
\section{Data}
The data sources for training punctuation retrieval tasks can be categorised into purely textual, and aligned audio with text.
The most common source used for training punctuation retrieval is the IWSLT dataset - many versions of which features transcripts from TED-talks and OpenSubtitles. This textual datasource is used to train a model that only uses lexical features for punctuation retrieval. \cite{jointlearningcorrbirnn} also used transcripts from the Intelligence squared debate show. \cite{medicalasr} used an internal medical speech transcript dataset to demonstrate the ability of their model to transfer to a domain with large vocabulary and data scarcity.
\textbf{Use of non-lexical features} \cite{multimodalsemi} used the audio and text data from the Fisher corpus in training a model using both acoustic and lexical features. \cite{yi2020adversarial} also made use of the Penn Treebank-3 dataset to train a POS tagger to improve the performance of the punctuation retrieval class. \cite{birnnattention} made use of Estonian text annotated with pause duration (e.g. <sil=0.030> for 30 milliseconds of silence) 



\section{Models}
\subsection{Embeddings}


\label{title_page}

\begin{enumerate}
    \item \textbf{Title.} Concise and informative. Titles are often used in information-retrieval systems. Avoid abbreviations and formulae where possible.
    \item \textbf{Author names and affiliations.} Where the family name may be ambiguous (e.g., a double name), please indicate this clearly. Present the authors' affiliation addresses (where the actual work was done) below the names. Indicate all affiliations with a lower-case superscript letter immediately after the author's name and in front of the appropriate address. Provide the full postal address of each affiliation, including the country name and, the e-mail address of each author.
    \item \textbf{Corresponding author.} Clearly indicate who will handle correspondence at all stages of refereeing and publication, also post-publication. Ensure that phone numbers (with country and area code) are provided in addition to the e-mail address and the complete postal address. Contact details must be kept up to date by the corresponding author.
    \item \textbf{Present/permanent address.} If an author has moved since the work described in the article was done, or was visiting at the time, a 'Present address' (or 'Permanent address') may be indicated as a footnote to that author's name. The address at which the author actually did the work must be retained as the main, affiliation address. Superscript Arabic numerals are used for such footnotes.
\end{enumerate}

\section{Reference style and reference list}
\label{reference_style}

\subsection{Reference Style}
Citations in the text should follow the referencing style used by the American Psychological Association. You are referred to the Publication Manual of the American Psychological Association, Sixth Edition, ISBN 978-1-4338-0561-5.  APA's in-text citations require the author's last name and the year of publication. You should cite publications in the text, for example, (Smith, 2020).  However, you should not use [Smith, 2020].

\subsection{Reference List}
References should be arranged first alphabetically by the surname of the first author followed by initials of the author's given name, and then further sorted chronologically if necessary. More than one reference from the same author(s) in the same year must be identified by the letters 'a', 'b', 'c', etc., placed after the year of publication. For example, Van der Geer, J., Hanraads, J. A. J., \& Lupton, R. A. (2010). The art of writing a scientific article. Journal of Scientific Communications, 163, 51-59. https://doi.org/10.1016/j.Sc.2010.00372. There should be no [1], [2], [3], etc in your references list. \cite{pandababa}


\section*{Acknowledgements}
Collate acknowledgements in a separate section at the end of the article before the references and do not, therefore, include them on the title page, as a footnote to the title or otherwise. List here those individuals who provided help during the research (e.g., providing language help, writing assistance or proof reading the article, etc.).

\bibliography{asr.bib}

\end{document}